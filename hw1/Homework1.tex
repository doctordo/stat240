\documentclass[11pt]{article}\usepackage[]{graphicx}\usepackage[]{color}
%% maxwidth is the original width if it is less than linewidth
%% otherwise use linewidth (to make sure the graphics do not exceed the margin)
\makeatletter
\def\maxwidth{ %
  \ifdim\Gin@nat@width>\linewidth
    \linewidth
  \else
    \Gin@nat@width
  \fi
}
\makeatother

\definecolor{fgcolor}{rgb}{0.345, 0.345, 0.345}
\newcommand{\hlnum}[1]{\textcolor[rgb]{0.686,0.059,0.569}{#1}}%
\newcommand{\hlstr}[1]{\textcolor[rgb]{0.192,0.494,0.8}{#1}}%
\newcommand{\hlcom}[1]{\textcolor[rgb]{0.678,0.584,0.686}{\textit{#1}}}%
\newcommand{\hlopt}[1]{\textcolor[rgb]{0,0,0}{#1}}%
\newcommand{\hlstd}[1]{\textcolor[rgb]{0.345,0.345,0.345}{#1}}%
\newcommand{\hlkwa}[1]{\textcolor[rgb]{0.161,0.373,0.58}{\textbf{#1}}}%
\newcommand{\hlkwb}[1]{\textcolor[rgb]{0.69,0.353,0.396}{#1}}%
\newcommand{\hlkwc}[1]{\textcolor[rgb]{0.333,0.667,0.333}{#1}}%
\newcommand{\hlkwd}[1]{\textcolor[rgb]{0.737,0.353,0.396}{\textbf{#1}}}%

\usepackage{framed}
\makeatletter
\newenvironment{kframe}{%
 \def\at@end@of@kframe{}%
 \ifinner\ifhmode%
  \def\at@end@of@kframe{\end{minipage}}%
  \begin{minipage}{\columnwidth}%
 \fi\fi%
 \def\FrameCommand##1{\hskip\@totalleftmargin \hskip-\fboxsep
 \colorbox{shadecolor}{##1}\hskip-\fboxsep
     % There is no \\@totalrightmargin, so:
     \hskip-\linewidth \hskip-\@totalleftmargin \hskip\columnwidth}%
 \MakeFramed {\advance\hsize-\width
   \@totalleftmargin\z@ \linewidth\hsize
   \@setminipage}}%
 {\par\unskip\endMakeFramed%
 \at@end@of@kframe}
\makeatother

\definecolor{shadecolor}{rgb}{.97, .97, .97}
\definecolor{messagecolor}{rgb}{0, 0, 0}
\definecolor{warningcolor}{rgb}{1, 0, 1}
\definecolor{errorcolor}{rgb}{1, 0, 0}
\newenvironment{knitrout}{}{} % an empty environment to be redefined in TeX

\usepackage{alltt}

\usepackage{fancyhdr}
\usepackage{amsmath}
\usepackage{float}
\usepackage[top=2in, bottom=1.5in, left=.5in, right=.5in]{geometry}
%\usepackage[font=small,labelfont=bf]{caption}



% Title.
% ------
\title{STAT 240 Homework 1}
\author{Rebecca Barter, Andrew Do and Kellie Ottoboni}
\IfFileExists{upquote.sty}{\usepackage{upquote}}{}
\begin{document}

\maketitle

 
 \section*{Question 1. Consider a box that contains 5 ``1'' tickets and 7 ``0'' tickets. Consider drawing 6 tickets from this box at random with replacement. Let $X_1, X_2, ..., X_6$ denote the 6 numbers you observe. Let $\bar{X}$ denote the average of the draws.}
 
 \subsection*{a) What is $E[\bar{X}]$?}
 
 
 \subsection*{b) What is $SE[\bar{X}]$? (R hint: Be careful whether the function ``sd'' divides by the square root of $n$ or $n - 1$)}
 
 \subsection*{c) Use R to simulate 100,000 values of $\bar{X}$. Produce a histogram of these values. (R hint: Use the function sample).}
 
 
 
 \subsection*{d) Let $z_1 = E[\bar{X}] + SE[\bar{X}], ~ z_2 = E[\bar{X}] + 2 \times SE[\bar{X}]$, etc. For $z_1, ..., z_4$ calculate $P(\bar{X} > z_i)$ in three ways:
 \begin{itemize}
 \item Exactly, using the binomial distribution. (Hint: It will be easier to work with the sample sum than the sample average. R hint: Use function pbinom)
 \item Estimated using the values from part (c)
 \item Using the normal approximation. Use the continuity correction. (R hint: pnorm)
 \end{itemize}
 Do the same for $z_{-4},...,z_{-1}$ but calcualte $P(\bar{X} < z_i)$ instead of $P(\bar{X} > z_i)$. Make a table of your results and comment briefly}
\begin{knitrout}
\definecolor{shadecolor}{rgb}{0.969, 0.969, 0.969}\color{fgcolor}\begin{figure}[H]

{\centering \includegraphics[width=\maxwidth]{figure/hist-1} 

}

\caption[Some silly histogram example]{Some silly histogram example}\label{fig:hist}
\end{figure}


\end{knitrout}

\subsection*{Repeat (a)-(d), this time sampling without replacement instead of with replacement. Use the hypergeometric distirbution instead fo the binomial distribuion (R hint: phyper)}
\end{document}



