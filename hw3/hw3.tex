
\documentclass[11pt]{article}
\usepackage{amsmath,amssymb}
\usepackage{graphicx}
\usepackage[margin=0.75in]{geometry}
\usepackage{bm}
\usepackage{color}
\usepackage[usenames,dvipsnames]{xcolor}
\usepackage{enumerate}

\setlength\parindent{0pt}
\newcommand{\pr}{P} % Generic probability
\newcommand{\var}{\textrm{Var}}
\newcommand{\cov}{\textrm{Cov}}


\title{STAT 240 Homework 3}
\author{Rebecca Barter, Andrew Do, and Kellie Ottoboni}
\date{March 13, 2015} % delete this line to display the current date

%%% BEGIN DOCUMENT
\begin{document}

\maketitle


\section*{1) Show that a permutation test based on $\bar{X}$ and a permutation test based on $t$ are equivalent when $m = n$} % Bec

\noindent Note that the $t$ statistic is defined by
$$t = \frac{\bar{X} - \bar{Y}}{\sqrt{\frac{Var(X)}{n} + \frac{Var(Y)}{m}}}$$
\noindent and that a permutation test based on $t$ involves
\begin{enumerate}
\item Fill a box with the observed data.
\item Draw a simple random sample of size $n$ and call if $X$, call the remaining elements in the box $Y$.
\item Compute $t$ as you would if $X$ and $Y$ were your original data. Call this $t^{*(1)}$.
\item Repeat steps 1-3 $L$ times to get $t^{*(2)}, t^{*(3)}, ..., t^{*(L)}$.
\item The distribution of the $t^{*(\ell)}$ approximates the true probability distribution of $t$ under the strong null. In particular, a (left-tail) $p$-value can be computed as
$$\frac{1}{L} \# \{t^{*(\ell)} \leq t\}$$
\end{enumerate}

\noindent We thus need to show that we can write
$$t^* = \frac{\bar{X}^* - \bar{Y}^*}{\sqrt{\frac{Var(X^*)}{n} + \frac{Var(Y^*)}{n}}}$$
\noindent in terms of $\bar{X}^*$ only.\\


\noindent We note, however, that if we simply write $A = \sum_{i=1}^n X_i + \sum_{i=1}^n Y_i   = \sum_{i=1}^n X_i^* + \sum_{i=1}^n Y_i^*$ to the the sum of all observations (which can be considered our new population from which we are drawing), we have that
$$\bar{X}^* - \bar{Y}^* = \left(1 + \frac{n}{n} \right) \bar{X}^* - \frac1n A = 2 \bar{X}^* - \frac1n A $$

\noindent Thus the RHS depends only on $\bar{X}^*$. Next, if we write $B =\sum_{i=1}^n X_i^2 + \sum_{i=1}^n Y_i^2 = \sum_{i=1}^n X_i^{*2} +\sum_{i=1}^n Y_i^{*2}$, then 

\begin{align*}
\frac{Var(X^*)}{n} + \frac{Var(Y^*)}{n} &= \frac{1}{n} \left[ \frac1n \sum_{i=1}^n X_i^{*2} - \frac{1}{n^2} \left( \sum_{i=1}^n X_i^*\right)^2 \right] + \frac{1}{n} \left[ \frac1n \sum_{i=1}^n Y_i^{*2} - \frac{1}{n^2} \left( \sum_{i=1}^n Y_i^*\right)^2 \right]\\
& = \left[ \frac{1}{n^2} \sum_{i=1}^n X_i^{*2} + \frac{1}{n^2} \sum_{i=1}^n Y_i^{*2} \right] - \frac1n\bar{X}^{*2} - \frac{1}{n^3} \left[A - n \bar{X}^* \right]^2\\
& = \frac{1}{n^4}\left[ n^2 \sum_{i=1}^n X_i^{*2} + n^2\left(B - \sum_{i=1}^n X_i^{*2}\right) \right] - \frac1n\bar{X}^{*2} - \frac{1}{n^3} \left[A - n \bar{X}^* \right]^2\\
& = \frac{B}{n^2} - \frac1n\bar{X}^{*2} - \frac{1}{n^3} \left[A - n \bar{X}^* \right]^2\\
\end{align*}

\noindent which depends only on $\bar{X}^{*}$. Thus we have shown that we can write $t^*$ in terms of $\bar{X}^*$ only, implying that $t$ and $\bar{X}$ are equivalent test statistics for a permutation test when $m  = n$.


\section*{2) Construct a hypothetical dataset (with at least 3 data points in treatment and at least 3 in control) for which the $p$-value of a permutation test based on $\bar{X}$ is smaller than the $p$-value of a permutation test based on $t$. Try to make the difference substantial.}


\section*{3) Construct a hypothetical dataset (with at least 3 data points in treatment and at least 3 in control) for which the $p$-value of a permutation test based on $t$ is smaller than the $p$-value of a permutation test based on $\bar{X}$. Try to make the difference substantial.}

\section*{4) Construct a hypothetical dataset (with at least 3 data points in treatment and at least 3 in control) for which the $p$-value of a permutation test based on $\bar{X}$ is smaller than the $p$-value of a standard $t$ test. Try to make the difference substantial.}

\end{document}