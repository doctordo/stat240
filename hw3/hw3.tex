
\documentclass[11pt]{article}
\usepackage{amsmath,amssymb}
\usepackage{graphicx}
\usepackage[margin=0.75in]{geometry}
\usepackage{bm}
\usepackage{color}
\usepackage[usenames,dvipsnames]{xcolor}
\usepackage{enumerate}

\setlength\parindent{0pt}
\newcommand{\pr}{P} % Generic probability
\newcommand{\var}{\textrm{Var}}
\newcommand{\cov}{\textrm{Cov}}


\title{STAT 240 Homework 3}
\author{Rebecca Barter, Andrew Do, and Kellie Ottoboni}
\date{March 13, 2015} % delete this line to display the current date

%%% BEGIN DOCUMENT
\begin{document}

\maketitle


\section*{1) Show that a permutation test based on $\bar{X}$ and a permutation test based on $t$ are equivalent when $m = n$} % Bec

\noindent Note that the $t$ statistic is defined by
$$t = \frac{\bar{X} - \bar{Y}}{\sqrt{\frac{Var(X)}{n} + \frac{Var(Y)}{m}}}$$
\noindent and that a permutation test based on $t$ involves
\begin{enumerate}
\item Fill a box with the observed data.
\item Draw a simple random sample of size $n$ and call if $X$, call the remaining elements in the box $Y$.
\item Compute $t$ as you would if $X$ and $Y$ were your original data. Call this $t^{*(1)}$.
\item Repeat steps 1-3 $L$ times to get $t^{*(2)}, t^{*(3)}, ..., t^{*(L)}$.
\item The distribution of the $t^{*(\ell)}$ approximates the true probability distribution of $t$ under the strong null. In particular, a (left-tail) $p$-value can be computed as
$$\frac{1}{L} \# \{t^{*(\ell)} \leq t\}$$
\end{enumerate}

\noindent We thus need to show that we can write
$$t^* = \frac{\bar{X}^* - \bar{Y}^*}{\sqrt{\frac{Var(X^*)}{n} + \frac{Var(Y^*)}{n}}}$$
\noindent in terms of $\bar{X}^*$ only.\\


\noindent We note, however, that if we simply write $A = \sum_{i=1}^n X_i + \sum_{i=1}^n Y_i   = \sum_{i=1}^n X_i^* + \sum_{i=1}^n Y_i^*$ to the the sum of all observations (which can be considered our new population from which we are drawing), we have that
$$\bar{X}^* - \bar{Y}^* = \left(1 + \frac{n}{n} \right) \bar{X}^* - \frac1n A = 2 \bar{X}^* - \frac1n A $$

\noindent Thus the RHS depends only on $\bar{X}^*$. Next, if we write $B =\sum_{i=1}^n X_i^2 + \sum_{i=1}^n Y_i^2 = \sum_{i=1}^n X_i^{*2} +\sum_{i=1}^n Y_i^{*2}$, then 

\begin{align*}
\frac{Var(X^*)}{n} + \frac{Var(Y^*)}{n} &= \frac{1}{n} \left[ \frac1n \sum_{i=1}^n X_i^{*2} - \frac{1}{n^2} \left( \sum_{i=1}^n X_i^*\right)^2 \right] + \frac{1}{n} \left[ \frac1n \sum_{i=1}^n Y_i^{*2} - \frac{1}{n^2} \left( \sum_{i=1}^n Y_i^*\right)^2 \right]\\
& = \left[ \frac{1}{n^2} \sum_{i=1}^n X_i^{*2} + \frac{1}{n^2} \sum_{i=1}^n Y_i^{*2} \right] - \frac1n\bar{X}^{*2} - \frac{1}{n^3} \left[A - n \bar{X}^* \right]^2\\
& = \frac{1}{n^4}\left[ n^2 \sum_{i=1}^n X_i^{*2} + n^2\left(B - \sum_{i=1}^n X_i^{*2}\right) \right] - \frac1n\bar{X}^{*2} - \frac{1}{n^3} \left[A - n \bar{X}^* \right]^2\\
& = \frac{B}{n^2} - \frac1n\bar{X}^{*2} - \frac{1}{n^3} \left[A - n \bar{X}^* \right]^2\\
\end{align*}

\noindent which depends only on $\bar{X}^{*}$. Thus we have shown that we can write $t^*$ in terms of $\bar{X}^*$ only, implying that $t$ and $\bar{X}$ are equivalent test statistics for a permutation test when $m  = n$.


\section*{2) Construct a hypothetical dataset (with at least 3 data points in treatment and at least 3 in control) for which the $p$-value of a permutation test based on $\bar{X}$ is smaller than the $p$-value of a permutation test based on $t$. Try to make the difference substantial.}


Suppose that the treatment group $X$ contains only $5$ observations, drawn from a Gaussian distribution with mean $0$ and standard deviation $20$.  In this example, the generated sample is

$$ X = \{ 10.9934, -16.8321,   0.6600,  10.4830, -34.5521\}$$

Let the control group $Y$ contains $100$ standard normal observations.  A situation like this might occur if a treatment is very expensive to administer and has variable effects.  \\

Suppose we want to test the strong null hypothesis using a two-sided test.  The observed difference in means is $\bar{X} - \bar{Y} = -5.7513$ and the observed t-statistic is $-0.6560$.  Using $10000$ simulations, the permutation test p-values for the two-sided test are

\begin{align*}
\pr(\lvert \bar{X} - \bar{Y} \rvert \geq 5.7513) &= 0.05 \\
\pr(\lvert t\rvert \geq 0.6560) &= 0.559
\end{align*}

The p-value of the permutation test based on $\bar{X}$ is smaller than the p-value of the permutation test based on $t$ because of the extreme noise and small sample size in the treatment group.  

\section*{3) Construct a hypothetical dataset (with at least 3 data points in treatment and at least 3 in control) for which the $p$-value of a permutation test based on $t$ is smaller than the $p$-value of a permutation test based on $\bar{X}$. Try to make the difference substantial.}


\section*{4) Construct a hypothetical dataset (with at least 3 data points in treatment and at least 3 in control) for which the $p$-value of a permutation test based on $\bar{X}$ is smaller than the $p$-value of a standard $t$ test. Try to make the difference substantial.}

Let the treatment group consist of the observations
$$ X = \{ -10, -9, \dots, 9, 10, 500, 1000, 2000, 5000\}$$

and the control group consist of the numbers $0$ through $5$, each appearing $5$ times.  Suppose that in the treatment group, the large observations are highly unusual.  If we were to exclude these, the two groups would have the same means.  We would like our test to be robust to these outliers and tell us that the difference in means between the treatment and control groups is not significantly different from $0$ under the strong null.  We will compare the strong null to the alternative hypothesis that the difference in means is greater than $0$.  \\

The observed difference in means is $\bar{X} - \bar{Y} = 337.5$ and the observed t-statistic is $1.5805$ ($df = 53$).  Using $10000$ simulations, the permutation test p-value for the mean is
$$\pr(\lvert \bar{X} - \bar{Y}\rvert \geq 337.5) = 0.0394 $$

The t-test p-value is
$$\pr(\lvert t_{53}\rvert \geq 1.5805) = 0.1200 $$


The p-value of the permutation test based on $\bar{X}$ is smaller than the p-value of the usual two-sample t-test based because of the extreme outliers in the treatment group.  When the data is highly skewed, the normality assumptions needed for the t-test are violated.
\end{document}