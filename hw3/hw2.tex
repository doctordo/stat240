
\documentclass[11pt]{article}
\usepackage{amsmath,amssymb}
\usepackage{graphicx}
\usepackage[margin=0.75in]{geometry}
\usepackage{bm}
\usepackage{color}
\usepackage[usenames,dvipsnames]{xcolor}
\usepackage{enumerate}

\setlength\parindent{0pt}
\newcommand{\pr}{P} % Generic probability
\newcommand{\var}{\textrm{Var}}
\newcommand{\cov}{\textrm{Cov}}


\title{STAT 240 Homework 3}
\author{Rebecca Barter, Andrew Do, and Kellie Ottoboni}
\date{March 13, 2015} % delete this line to display the current date

%%% BEGIN DOCUMENT
\begin{document}

\maketitle


\section*{1) Show that a permutation test based on $\bar{X}$ and a permutation test based on $t$ are equivalent when $m = n$}

\section*{2) Construct a hypothetical dataset (with at least 3 data points in treatment and at least 3 in control) for which the $p$-value of a permutation test based on $\bar{X}$ is smaller than the $p$-value of a permutation test based on $t$. Try to make the difference substantial.}


\section*{3) Construct a hypothetical dataset (with at least 3 data points in treatment and at least 3 in control) for which the $p$-value of a permutation test based on $t$ is smaller than the $p$-value of a permutation test based on $\bar{X}$. Try to make the difference substantial.}

\section*{4) Construct a hypothetical dataset (with at least 3 data points in treatment and at least 3 in control) for which the $p$-value of a permutation test based on $\bar{X}$ is smaller than the $p$-value of a standard $t$ test. Try to make the difference substantial.}

\end{document}