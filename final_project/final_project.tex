\documentclass[11pt]{article}

\usepackage{fancyhdr}
\usepackage{amsmath}
\usepackage{graphicx}
\usepackage{float}
\usepackage[top=1in, bottom=1in, left=.5in, right=.5in]{geometry}
%\usepackage[font=small,labelfont=bf]{caption}

% Title.
% ------
\title{STAT 240 Final Project}
\author{Rebecca Barter, Andrew Do and Kellie Ottoboni}



\begin{document}

\maketitle

 \section{Introduction}
 
 
 \section{The data}
 
 
 
 \section{Analysis}
 \subsection{Value-added}
 
 \subsection{Stratified permutation tests}
 
 
 \subsection{Regression discontinuity}
Having seen that tracking has an extreme effect on those students at the high and low ends of the ability spectrum, we now introduce the regression discontinuity approach to test the effect of tracking on those in the middle. In particular we focus on the students whose initial test score was near the cutoff (median initial test score). 
 
 
 \section{Results}
 
\end{document}
